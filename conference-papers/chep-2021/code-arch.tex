\hypertarget{code-architecture}{%
\subsection{Code architecture}\label{code-architecture}}

In the short term, Celeritas is designed as a standalone application
that transport particles exclusively on device. To support robust and
rapid unit testing, its components are designed to run natively in C++
on traditional CPUs regardless of whether CUDA is available for
on-device execution.

Like other GPU-enabled Monte Carlo transport codes such as Shift, the
low-level component code used by transport kernels is designed so that
each particle track corresponds to a single thread, since particle
tracks once created are independent of each other. There is therefore
essentially no cooperation between individual threads, facilitating the
dual host/device annotation of most of Celeritas. The allocation of
secondary particles and the initialization of new tracks from these
secondaries both require CUDA-specific programming, but those components
are encapsulated so that both host and device code can safely construct
secondaries.

To support parallelizing our initial development over several team
members, and to facilitate refactoring and performance testing of code,
Celeritas uses a highly modular programming approach based on
composition rather than inheritance. As much as possible, each major
code component is built of numerous smaller components and interfaces
with as few other components as possible. The interfaces with other
components are furthermore
