%%---------------------------------------------------------------------------%%
% integration.tex
%%---------------------------------------------------------------------------%%
\section{Integration with \ac{hep} workflows}

\celcomment{SCT: not happy about this paragraph yet}

\ac{doe} \acp{lcf} are planned to be part of \ac{hep} workflows by the
scientific community, with their use ranging from simulation and reconstruction
to \ac{ai} methods \cite{hep-network-requirements}. While the Cosmic Frontier is
already taking advantage of facilities such as \ac{alcf}, \ac{nersc}, and
\ac{olcf}, the Energy and Intensity Frontiers have less clear integrations, as
the vast majority of tools in these workflows are \ac{cpu}-based. In this
scenario, \celeritas aims close the gap between \ac{hep} distributed computing
networks and \acp{lcf} networks by providing three different routes
(Fig.~\ref{fig:celeritas-hep-workflows}). These workflows  will enable
\celeritas:
%%
\begin{enumerate}[itemsep=0pt, label=(\alph*)]
  \item to accelerate standard \ac{hep} detector simulation workflows built on
    Geant4 by offloading \ac{em} particle showers to \acp{gpu} using a new
    \acceleritas library (\S~\ref{sec:acceleritas});
  \item to run complete end-to-end detector simulations with comprehensive
    \ac{sm} physics at the \acp{lcf} (\S~\ref{sec:end-to-end}); and
  \item to generate high-resolution detector responses as training data for
    \ac{ai} networks to be deployed at experimental facilities as software
    triggers and surrogate models (\S~\ref{sec:celeritas-ai}).
\end{enumerate}
%%
\begin{figure}
    \centering
    \includegraphics[width=\textwidth]{figs/celeritas_integration-all}
    \caption{Proposed \acs{hep} integration workflows for (a) \acceleritas, (b)
    end-to-end \celeritas, and (c) \celeritas for \acs{ai}.}
    \label{fig:celeritas-hep-workflows}
\end{figure}
%%

% - - - - - - - - - - - - - - - - - - - - - - - - - - - - - - - - - - - - - - %
\subsection{Acceleritas}
\label{sec:acceleritas}

% - - - - - - - - - - - - - - - - - - - - - - - - - - - - - - - - - - - - - - %
\subsection{End-to-end \celeritas}
\label{sec:end-to-end}

% - - - - - - - - - - - - - - - - - - - - - - - - - - - - - - - - - - - - - - %
\subsection{\celeritas for \ac{ai}}
\label{sec:celeritas-ai}

% - - - - - - - - - - - - - - - - - - - - - - - - - - - - - - - - - - - - - - %
\subsection{Integration challenges}

The heterogeneity of \ac{hep} computing workflows, associated with the volume of
data produced by each experiment, pose a long list of challenges that need to be
overcome in order to make \celeritas a viable option. We outline here the most pressing ones, along with mitigation plans. 
%%
\begin{enumerate}[itemsep=0pt]
  \item Simulation inputs must be flexible enough to encompass Energy and
    Intensity Frontier needs. This includes user-defined geometry, physics,
    events, secondary particle cutoff thresholds, and sensitive detector scoring
    regions.
  \item Output data, which entails \ac{mc} particle history and detector
    scoring, should be flexible enough to make it compatible to experimental
    workflows, while maximizing \ac{io} efficiency.
  \item End-user interface must be seamless enough such that the balance between
    performance gain and the work needed to adapt experimental computing
    workflows justify the adoption.
  \item Domestic and international networking between \acp{lcf} and \ac{hep}
    computing centers can lead to large data migration which can result in
    network congestions and suboptimal resource usage.  The \ac{lhc}'s ``any
    data, anywhere, anytime'' model \cite{hep-network-requirements} might need
    special attention.
\end{enumerate}

These main challenges, among other topics, will be discussed and addressed via a
\celeritas \emph{User Council}. This council will be formed by members of
different \ac{hep} experiments on both Energy and Intensity Frontiers and will
advise the \celeritas team, so that the code develops focusing on experimental
compatibility.
