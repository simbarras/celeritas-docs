%%---------------------------------------------------------------------------%%
% integration.tex
%%---------------------------------------------------------------------------%%
\section{Integration with HEP workflows}

To be successfully adopted by the \ac{hep} community, \celeritas must be
integrated into established experimental detector simulation workflows and must
provide new workflows incorporating \ac{doe}'s \acp{lcf}. In
Fig.~\ref{fig:celeritas-hep-workflows} we present three interfaces that will
enable \celeritas:
%%
\begin{enumerate}[itemsep=0pt, label=(\alph*)]
  \item to accelerate standard \ac{hep} detector simulation workflows built on
    Geant4 by offloading \ac{em} particle showers to \acp{gpu} using a new
    \acceleritas library described below (\S~\ref{sec:acceleritas}),
  \item to run complete end-to-end detector simulations with comprehensive
    \ac{sm} physics at the \acp{lcf} (\S~\ref{sec:end-to-end}), and
  \item to generate high-resolution detector responses as training data for
    \ac{ai} networks to be deployed at experimental facilities as software
    triggers and surrogate models (\S~\ref{sec:celeritas-ai}).
\end{enumerate}
%%
\begin{figure}
    \centering
    \includegraphics[width=\textwidth]{figs/celeritas_integration-all}
    \caption{Proposed \acs{hep} integration workflows for (a) \acceleritas, (b)
    end-to-end \celeritas, and (c) \celeritas for \acs{ai}.}
    \label{fig:celeritas-hep-workflows}
\end{figure}
%%

\subsection{Acceleritas}
\label{sec:acceleritas}

\subsection{End-to-end \celeritas}
\label{sec:end-to-end}


\subsection{\celeritas for AI}
\label{sec:celeritas-ai}
