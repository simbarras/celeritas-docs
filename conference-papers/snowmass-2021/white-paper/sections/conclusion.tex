%%---------------------------------------------------------------------------%%
% conclusion.tex
%%---------------------------------------------------------------------------%%
\section{Conclusion}

We have presented \celeritas, a project designed to provide high-fidelity
\ac{mc} detector simulation transport capabilities on current and next-gen
\ac{gpu} architectures. Over the next years, comprehensive \ac{sm} physics
including \ac{em}, hadronic, and decay physics, along with \ac{em} fields will
be implemented. \celeritas' completion is planned to happen before the beginning
of \acs{hllhc}'s Run 4. This includes a fully fledged \ac{io} system, along with
its integration with ROOT, becoming a viable option to alleviate the impending
\ac{mc} computing requirements of the next generation of \ac{hep} experiments.
With current preliminary results showing two orders of magnitude speedups
compared to single-threaded Geant4 executions, \celeritas has the potential to
bring the massive computing power provided by the \ac{doe} \acp{lcf} into
\ac{hep} workflows.

The enabling technologies that will allow interfacing between end-to-end
simulations performed at the \acp{lcf} and experimental computing centers will
yield other long-term benefits for interactions between the \ac{doe} \acp{lcf}
and experimental compute nodes. The planned \ac{io} capabilities that need to be
developed for \celeritas will provide full interoperability between data
produced at the \acp{lcf} and ROOT, which might be benefical for other \ac{hep}
code bases.

The \celeritas project foreshadows proposed efforts in federated computing, in
which the \acp{lcf} interact directly with compute nodes at experimental
facilities to provide optimal use of compute resources.  In this case, we
envision a scenario where expensive \ac{mc} simulations are performed at the
\acp{lcf} and simulation output is communicated directly to experimental
\ac{hep} distributed computing networks for reconstruction and analysis.  The
successful execution of this project can therefore be the genesis for a host of
technological advancements in the use of \ac{hpc} to enable more science output
in all three \ac{doe} \ac{hep} frontiers.
