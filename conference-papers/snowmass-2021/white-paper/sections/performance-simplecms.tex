%%---------------------------------------------------------------------------%%
% performance.tex
%%---------------------------------------------------------------------------%%
\section{\celeritas performance}

Given the range of detector geometry complexities, primary event types, and
amount of output, there is no general case that provides a simple performance
number such as a simulation rate in events per second. Therefore, our ultimate
performance metric will be based on a set of real-world \acs{hep} detector
workflow use cases. \celeritas must run the targeted models at the same fidelity
as the current state of the art but in much less time. In the next sections,  we
will present the two \acp{fom} that will be used to measure success, along with
a few preliminary performance results comparing \celeritas with current
state-of-the-art \ac{mc} codes.

% - - - - - - - - - - - - - - - - - - - - - - - - - - - - - - - - - - - - - - %
\subsection{Performance metrics}

On \ac{lcf} hardware for the next decade, \acp{gpu} will provide the bulk of the
compute capability, so one critical performance metric is the ratio of the
runtime of \celeritas using \acp{gpu} to using only the \acp{cpu} of a given
machine. This \ac{fom} represents the ability of \celeritas to effectively use
the hardware that is available and must be sufficiently high to justify running
on \ac{lcf} resources. We will target a factor of $160\times$ for a single
\ac{gpu} to a single \ac{cpu}, which is the relative \ac{gpu}/\ac{cpu}
performance of the Shift \ac{mc} transport code
\cite{hamilton_continuous-energy_2019}.

A second performance metric is critical to the programmatic viability of
\celeritas in the broader \ac{hep} community: the work done for the same amount
of cost in power consumption and hardware, using Geant4 on \ac{cpu} as a
baseline. This \ac{fom} is the motivation for \ac{hep} workflows to adapt to
using \celeritas for \ac{mc} transport. The Geant4 team supposes that a factor
of two speedup resulting from ``adiabatic improvements'' to their code is not
outside the realm of possibility \cite{marc_verderi_geant4_2021}, so a $2
\times$ cost improvement of \celeritas runtime over Geant4 runtime is our second
target. Assuming that electricity consumption (and waste heat disposal) are the
primary constraints for independent \ac{hep} computing centers, this factor
should be evaluated by comparing the performance of Geant4 on \acp{cpu} with a
comparable power requirement as \celeritas on \acp{gpu}. At the present time,
for example, that comparison might be for an AMD 3rd Gen EPYC Processor power
(\SI{280}{\watt} for 64 cores) to a PCIe \nvidia A100 (\SI{250}{\watt} for 108
symmetric multiprocessors).

Meeting these \acp{fom} with Geant4-equivalent physics capabilities and
providing a solid user interface and \ac{io} integration to \ac{hep} workflows
will be the ultimate confirmation that \celeritas is a viable option for
\ac{hep} experiments. At this point \celeritas will be ideal for execution on
\ac{doe} \acp{lcf} and will be sufficiently fast on its own merits to motivate
independent adoption on capacity systems.

% - - - - - - - - - - - - - - - - - - - - - - - - - - - - - - - - - - - - - - %
\subsection{Preliminary results}

The \celeritas team will develop a series of internal test problems to validate
every aspect of the code, including physics, geometry, \ac{em} fields, and
\ac{io}. Here, we present a small set of preliminary results comparing
\celeritas against Geant4 using our most recently implemented high-level
verification problem. This test evaluates the current code as a whole, including
features such as multiple materials, secondary production, and energy cutoffs
with the full array of currently implemented physics processes for $\gamma$,
$e^-$, and $e^+$ (Table~\ref{tab:em-physics}). The test geometry is an idealized
coarse representation of the main components of the \ac{cms} detector,
comprising six cylindrical shells centered about the $z$ axis up to
\SI{375}{\centi\meter}. Each cylinder and shell is composed of a single element:
vacuum, Si, Pb, C, Ti, and Fe. 

The first preliminary results are on a compute node that uses the same \acp{gpu}
as \ac{olcf}'s Summit: it has Intel Xeon Gold 5218 (Cascade Lake) \acp{cpu}
running at 2.3GHz (2 sockets of 16 cores with 2 hardware threads per core, with
\SI{188}{\giga\byte} total memory), and \nvidia V100 \acp{gpu} (each with 80
symmetric multiprocessors, 64 CUDA cores per multiprocessor, and
\SI{16}{\giga\byte} memory). The simulation input is an isotropic
\SI{1}{\giga\electronvolt} photon point source at the origin, composed of 100
events each with \num{1000} photon primaries.

The initial physics validation tallies the cumulative energy deposition over the
$z$ axis (Fig.~\ref{fig:physics-results:edep}), and the distribution of number
of steps per track, separated by particle type
(Fig.~\ref{fig:physics-results:steps}). These show the two \ac{mc} codes to be
in agreement within statistical uncertainties.
%%
\begin{figure}
  \centering%
  \begin{subfigure}{\textwidth/2}%
    \includegraphics[height=2.6in]{figs/cumulative-z}%%
    \caption{Cumulative energy deposition over the $z$ axis.}%
    \label{fig:physics-results:edep}
  \end{subfigure}%
  \begin{subfigure}{\textwidth/2}%
    \includegraphics[height=2.6in]{figs/steps-per-track}%
    \caption{Steps per track, separated by particle type.}%
    \label{fig:physics-results:steps}
  \end{subfigure}
  \caption{Comparison between Geant4 and \celeritas simulation runs. Results are
  in agreement within the statistical errors.}
  \label{fig:physics-results}
\end{figure}
%%
The performance numbers can only provide a baseline prediction for the eventual
performance of \celeritas, as it has only a small subset of its eventual
capabilities. Output routines are disabled in order to control for the potential
latency of disk \ac{io}, which will be improved in the future. Because the
current \celeritas application uses the same transport loop algorithm for both
\ac{cpu} and \ac{gpu} implementations, where each step is divided into numerous
kernels with many synchronization points, many tracks must be ``in flight''
simultaneously to amortize the overhead of thread synchronization on \ac{cpu}.
On \ac{gpu}, one thread corresponds to a single track, but the \ac{cpu} can
transport an arbitrary number of tracks per physical core. This problem uses
\num{1024} total tracks regardless of core count on \ac{cpu}. For the results
presented in Fig.~\ref{fig:celeritas-performance}, only one \ac{cpu} socket (up
to 16 physical cores) and a single \ac{gpu} are used. Peak \ac{gpu} performance
is reached with about a million simultaneous tracks. At this peak, the \ac{gpu}
was about $30 \times$ faster than a 16-core run of the OpenMP version of
\celeritas,  $170 \times$ faster than the single-core \ac{cpu} \celeritas
runtime, and about $240 \times$ faster than the single-core \ac{cpu} Geant4
runtime.

\begin{figure}
    \centering
    \includegraphics{figs/performance.pdf}
    \caption{Performance of Celeritas and Geant4 on \acs{cpu} and \acs{gpu}.
    Higher values are better.}
    \label{fig:celeritas-performance}
\end{figure}

These initial results, where a single \nvidia V100 is equivalent to 240 Intel
Xeon Gold 5218 \ac{cpu} cores at 2.3 GHz, are surprisingly good for an initial
unoptimized simulation of a realistic problem with multiple realistic physics,
but they do come with caveats:
\begin{itemize}[itemsep=0pt]
  \item This demonstration version of \celeritas supports only a single element
    per material, and the demonstration problem is hardwired to have no magnetic
    field. Including those features will slow down the simulation even if they
    are unused.
  \item The \ac{prng} used in the \celeritas demonstration ({XORWOW}) is faster
    and less statistically random than the \ac{prng} in Geant4 ({MixMax}).
  \item The current \celeritas algorithm and data structures are designed for
    massive \ac{gpu} parallelism and are not optimal for \ac{cpu} parallelism,
    especially with a small thread count.
  \item Aside from early performance analysis on a small subset of the present
    code base in \cite{johnson_2021}, \emph{no optimization work has been
    performed} for this demonstration, so \celeritas may become faster.
  \item The simulation results are reproducible with the same number of threads,
    but individual track IDs (used for ``MC truth'' debugging analysis) are
    arbitrary.
  \item As with all other \ac{gpu} applications, good performance requires
    saturating the \ac{gpu} with work to do, hiding latency for memory accesses
    and amortizing the launch cost and \ac{cpu} overhead. Experimental workflows
    may need to batch multiple events together to achieve peak \ac{gpu}
    performance.
\end{itemize}
Taking these considerations together, we extrapolate these initial results to be
representative of expected performance on a fully featured \celeritas \ac{em}
simulation.

Figure~\ref{fig:celeritas-steps} dives into the performance characteristics of
the many-threaded \ac{gpu} execution by enabling extra diagnostics and detailed
timing.
%%
\begin{figure}
    \centering%
    \begin{subfigure}{\textwidth/2}%
      \includegraphics[height=2.6in]{figs/alive-steps.pdf}%%
      \caption{Track and secondaries}%
      \label{fig:gpu-thread-active}%
    \end{subfigure}%
    \begin{subfigure}{\textwidth/2}%
      \includegraphics[height=2.5in]{figs/time-per-step.pdf}%
      \caption{Performance per step}%
      \label{fig:gpu-thread-time}%
    \end{subfigure}%
    \caption{Characterization on \acs{gpu} of (a) number of tracks in-flight and
    queued, and (b) time taken for each parallel step.}%
    \label{fig:celeritas-steps}%
\end{figure}
%%
As plotted in Fig.~\ref{fig:gpu-thread-active}, the behavior of the tracks in
flight (blue lines, left axis) and queued secondary initializers (red line,
right axis) determines the maximum thread count in the current implementation,
which is limited by the memory requirements of the initializers: the current
peak of \num{2.5e7} secondaries must at present be below a user-selected
preallocated secondary capacity. For the first ${\sim}10$ steps the total
particle count rises precipitously; but a lower active track count will tend to
complete the transport of low-energy secondaries before starting to transport
the high-energy particles that will increase the total number of tracks in
flight. By preferentially selecting lower-energy particles for tracking,
algorithmic improvements could reduce the total capacity requirement. Additional
code improvements should allow the secondary capacity to reallocate dynamically
and even buffer secondaries in \ac{cpu} memory or \ac{nvm}, maximizing the
number of active tracks and taking full advantage of massive \ac{gpu}
parallelism.

The timing breakdown in Fig.~\ref{fig:gpu-thread-time} zooms in on time
requirements of the steep rise and slow fall of active tracks. The simulation
required seven step iterations to saturate the \ac{gpu} with $2^{21}$ (about two
million) active tracks from the \num{100000} initial primaries. The increase in
time from the initial step (all tracks encounter a boundary since they are born
in a vacuum) to the next step (many tracks interact in matter) is the cost of
running nontrivial physics interaction kernels. Almost a third of the total
iterations are to ``drain'' the pipeline of secondaries. Notably, at the end of
the transport loop, the minimum step iteration time is still more than 10\% of
the maximum step time even with a handful of active tracks. This reiterates the
importance of exposing parallelism for \ac{gpu} performance.

With these initial results we have established a rough projection of the code's
eventual performance and have begun to characterize its behavior on device as a
function of workload. We will continue to use performance results to inform both
low-level code optimizations (to efficiently simulate many tracks in parallel)
and high-level workflow integration (to ensure enough tracks are being simulated
simultaneously).