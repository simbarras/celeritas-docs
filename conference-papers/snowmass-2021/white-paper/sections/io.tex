%%---------------------------------------------------------------------------%%
% io.tex
%%---------------------------------------------------------------------------%%
\section{User-interface and \ac{io}}

The key challenge in providing user interfaces is developing a framework that
supports integration into existing experimental \ac{mc} simulation sequences
that \ipl{1} provides the minimum required code barrier for incorporation and
\ipl{2} preserves performance.  Existing experimental frameworks are built on
the Geant4 toolkit, which provides user actions that allow users to control the
program and data flow at the level of run, event, track, and step. In Geant4
scoring is done using dedicated stepping actions in which information from the
sensitive detector volumes is accessible through callback semantics into these
parts of the simulation.  This approach provides great user flexibility at the
cost of higher computational overhead and increased system complexity.
Furthermore, callback functions will not work in accelerator code because it is
not possible to call host functions in the middle of device kernel execution.

\celeritas will not operate as a toolkit as Geant4 does, since this would leave
many implementation decisions to the end-user, hindering performance. Therefore,
to address the purely technical challenges of supporting experimental workflows,
we will implement an \ac{api} through which clients can specify geometric
regions for scoring and \ac{mc} particle data. Using this \ac{api} the desired
scoring data can be processed on the host at runtime, and the necessary data
fields for tallies can then be configured for execution in kernel code on the
device.  The most efficient interface would fully occupy the device by executing
many events concurrently.  However, this is not the way that experimental
workflows are currently configured, in which events are executed independently
on each thread.

As \ac{hpc} evolved to make use of heterogeneous architectures, \ac{io} became
one of the critical performance bottlenecks of many \ac{hpc} applications, being
a main concern for \celeritas. \ac{hep} detector simulations produce large
volumes of data, with many millions of particle tracks and detector scoring
regions having to be recorded. Thus, the need to optimize the data movement
between host and device and manage parallel \ac{io} requests is paramount.
Furthermore, to be compatible with \ac{hep} experimental workflows, \celeritas
needs to be integrated with ROOT. To mitigate these problems, we will cross
collaborate with \ac{doe}'s \ac{rapids} team to make \ac{adios} the internal
\ac{io} \ac{api} of \celeritas, as it is highly optimized for heterogeneous
architectures. The ultimate goal is find optimal strategies to mitigate \ac{io}
performance issues, and integrate \ac{adios} with ROOT for full interoperability
with \ac{hep} workflows. This colaboration with \ac{rapids} will also explore
data visualization tools and event filtering, allowing users to visualize,
validate, and debug the generated data before launching production campaigns.
