%%---------------------------------------------------------------------------%%
% physics.tex
%%---------------------------------------------------------------------------%%
\section{Physics}

The key physics component in \celeritas is a ``process,'' which defines an
observed physical phenomenon such as the photoelectric effect or bremsstrahlung.
Each process is implemented as one or more models that each mathematically
describe or approximate the process in a given energy regime.

The initial implementation in \celeritas targets \ac{em} physics between
\SI{100}{eV} and \SI{100}{TeV} for photons, electrons, and positrons. This
minimal set of capabilities, with physical processes and associated numerical
models itemized in Table~\ref{tab:em-physics}, is necessary to generate
realistic simulations of \ac{em} showers and demonstrate key characteristics of
a full-featured transport loop.
%%
\begin{table}
  \caption{Current status of \celeritas \acs{em} physics. The initial
  implementation ($\gamma$ and $e^\pm$) is almost complete, and muon physics is
  in its initial stage. Particle symbols are defined in
  \textcite{tanabashi_review_2018}.}
  \label{tab:em-physics}
  \centering
  \begin{tabular}{clll}\toprule Particle & Process & Model(s) & Status\\
    \midrule
    %% gamma
    \multirow{4}{*}{$\gamma$}
    & photon conversion & Bethe--Heitler & implemented\\
    & Compton scattering & Klein--Nishina & verified\\
    & photoelectric effect & Livermore & implemented\\
    & Rayleigh scattering & Livermore & implemented\\
    \midrule
    %% e^+-
    \multirow{4}{*}{$e^\pm$}
    & ionization & M\o{}ller--Bhabha & implemented\\
    & bremsstrahlung & Seltzer--Berger, relativistic & implemented\\
    & pair annihilation & EPlusGG & implemented\\
    & multiple scattering & Urban, WentzelVI & in progress\\
    \midrule
    $\mu^\pm$ & muon bremsstrahlung & Muon Bremsstrahlung & implemented\\
    \bottomrule
  \end{tabular}
\end{table}
%%
These already implemented characteristics include \ipl{1}
material-dependent physical properties, \ipl{2} continuous slowing down in
matter for charged particles, \ipl{3} selecting discrete interactions among
competing processes, \ipl{4} scattering or absorbing particles during an
interaction, \ipl{5} emitting secondary particles, and \ipl{6} applying energy
cutoffs to cull low-energy photons and electrons.

The physics implementation in \celeritas focuses on maximizing work done in
parallel. For example, all particle types use tabulated discrete interaction
cross sections calculated simultaneously in a single kernel. The primary
deviation from this rule is that each model of a discrete process launches an
independent kernel that applies only to tracks undergoing an interaction with
that process. This set of kernel launches is performed polymorphically from
\ac{cpu} host code, allowing arbitrary noninvasive extensions to \celeritas
physics.

In order to meet the detector simulation requirements for \acs{hep} experiments,
\celeritas physics will be expanded from its initial \ac{em} prototype to a full
set of particles with decay and hadronic physics.  A complete list of the
required physics processes and particles is shown in
Table~\ref{tab:proposed-physics}, where only processes are explicit as model
separation will be determined based on performance and/or code maintainability.
%%
\begin{table}
  \caption{Proposed physics development in \celeritas.}
  \label{tab:proposed-physics}
  \centering
  \begin{tabular}{llll}
    \toprule
    Physics & Process & Particle(s)\\
    \midrule
    %% EM
    \multirow{10}{*}{\acs{em}} & photon conversion & $\gamma$\\
    & pair annihilation & $e^\pm$\\
    & photoelectric effect& $\gamma$\\
    & ionization & charged leptons, hadrons, and ions\\
    & bremsstrahlung & charged leptons and hadrons\\
    & Rayleigh scattering & $\gamma$\\
    & Compton scattering & $\gamma$\\
    & Coulomb scattering & charged leptons, hadrons\\
    & multiple scattering & charged leptons, hadrons\\
    & continuous energy loss & charged leptons, hadrons, and ions\\
    \midrule
    %% Decay
    \multirow{3}{*}{Decay}
    & two body decay & $\mu^\pm$, $\tau^\pm$, hadrons\\
    & three body decay & $\mu^\pm$, $\tau^\pm$, hadrons\\
    & n-body decay & $\mu^\pm$, $\tau^\pm$, hadrons\\
    \midrule
    %% Hadronic
    \multirow{6}{*}{Hadronic}
    & photon-nucleus & $\gamma$ \\
    & lepton-nucleus & leptons \\
    & nucleon-nucleon & $p$, $n$\\
    & hadron-nucleon & hadrons\\
    & hadron-nucleus & hadrons\\
    & nucleus-nucleus & hadrons\\
    \bottomrule
  \end{tabular}
\end{table}
%%
