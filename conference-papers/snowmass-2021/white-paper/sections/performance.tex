%%---------------------------------------------------------------------------%%
% performance.tex
%%---------------------------------------------------------------------------%%
\section{\celeritas performance}

Given the range of detector geometry complexities, primary event types, and
amount of output, there is no general case that provides a simple performance
number such as a simulation rate in events per second. Therefore, our ultimate
performance metric will be based on a set of real-world \acs{hep} detector
workflow use cases. \celeritas must run the targeted models at the same fidelity
as the current state-of-the-art but in much less time. In the next sections,  we
will present the two \acp{fom} that will be used to measure success, along with
a few preliminary performance results comparing \celeritas with current
state-of-the-art \ac{mc} codes.

% - - - - - - - - - - - - - - - - - - - - - - - - - - - - - - - - - - - - - - %
\subsection{Performance metrics}

On \ac{lcf} hardware for the next decade, \acp{gpu} will provide the bulk of the
compute capability, so one critical performance metric is the ratio of the
runtime of \celeritas using \acp{gpu} to using only the \acp{cpu} of a given
machine. This \ac{fom} represents the ability of \celeritas to effectively use
the hardware that is available and must be sufficiently high to justify running
on \ac{lcf} resources. We will target a factor of $160\times$ for a single
\ac{gpu} to a single \ac{cpu}, which is the relative \ac{gpu}/\ac{cpu}
performance of the Shift \ac{mc} transport code
\cite{hamilton_continuous-energy_2019}.

A second performance metric is critical to the programmatic viability of
\celeritas in the broader \ac{hep} community: the work done for the same amount
of cost in power consumption and hardware, using Geant4 on \ac{cpu} as a
baseline. This \ac{fom} is the motivation for \ac{hep} workflows to adapt to
using \celeritas for \ac{mc} transport. The Geant4 team supposes that a factor
of two speedup resulting from ``adiabatic improvements'' to their code is not
outside the realm of possibility \cite{marc_verderi_geant4_2021}, so a $2
\times$ cost improvement of \celeritas runtime over Geant4 runtime is our second
target. Assuming that electricity consumption (and waste heat disposal) are the
primary constraints for independent \ac{hep} computing centers, this factor
should be evaluated by comparing the performance of Geant4 on \acp{cpu} with a
comparable power requirement as \celeritas on \acp{gpu}. At the present time,
for example, that comparison might be for an AMD 3rd Gen EPYC Processor power
(\SI{280}{\watt} for 64 cores) to a PCIe \nvidia A100 (\SI{250}{\watt} for 108
symmetric multiprocessors).

Meeting these \acp{fom} with Geant4-equivalent physics capabilities and
providing a solid user interface and \ac{io} integration to \ac{hep} workflows
will be the ultimate confirmation that \celeritas is a viable option for
\ac{hep} experiments. At this point \celeritas will be ideal for execution on
\ac{doe} \acp{lcf} and will be sufficiently fast on its own merits to motivate
independent adoption on capacity systems.

% - - - - - - - - - - - - - - - - - - - - - - - - - - - - - - - - - - - - - - %
\subsection{Preliminary results}

The \celeritas team is developing a series of internal test problems to validate
every aspect of the code, including physics, geometry, \ac{em} fields, and
\ac{io}. Additionaly, we are participating with the \acs{adept} project on a
series of common physics and performance benchmarks. Here, we present a small
set of preliminary results comparing \celeritas against Geant4 and \acs{adept}
using the \emph{TestEm3} benchmark test problem, available in \acs{adept}'s
repository\footnote {
    \url{https://github.com/apt-sim/AdePT}
}. This problem's geometry is composed of 50 rectangular layers, each
containing a Pb absorber and a liquid Ar gap.

In order to run a comparison between the three codes, we ported the
\acs{adept}'s implementation of the \emph{TestEm3} geometry to a Geant4
application, which in turn exported the geometry as a \acs{gdml} file that is
the current default input used by \celeritas. All three applications are running
the same physics models with the same secondary production thresholds.
Nevertheless, there are a few main caveats:
\begin{itemize}[itemsep=0pt]
    \item \celeritas and \acs{adept} use different \acp{prng};
    \item Geant4 uses its own navigation system, whereas \celeritas and
    \acs{adept} rely on \acs{vecgeom};
    \item \acs{adept} runs one event at a time, while celeritas loads and runs
    all primaries concurrently;
\end{itemize}

The results are produced using a particle-gun type input with \num{100000}, 10
GeV, electron primaries placed before the first layer. Their momentum direction
is fixed and pointed perpendicularly towards the layers. The compute node uses
the same \acp{gpu} as \ac{olcf}'s Summit: it has Intel Xeon Gold 5218 (Cascade
Lake) \acp{cpu} running at 2.3GHz (2 sockets of 16 cores with 2 hardware threads
per core, with \SI{188}{\giga\byte} total memory), and \nvidia V100 \acp{gpu}
(each with 80 symmetric multiprocessors, 64 CUDA cores per multiprocessor, and
\SI{16}{\giga\byte} memory). The current physics equivalency relies only on the
mean deposited energy per layer (Fig.~\ref{fig:testem3-edep}), which shows good
agreement between \acs{adept} and Geant4, with \celeritas being within a few
percent. Although the reason for this discrepancy still has to be fully
understood, it is likely caused by \celeritas' current linear interpolation of
energy-loss cross-section data. Geant's documentation explicitly states that for
accurate physics results, higher-order interpolations methods, such as
splined-based or log-log, are needed. The performance is measured in events per
second---where an event here is one primary particle---and presented in
Table~\ref{tab:testem3-performance}. \celeritas shows a speedup factor of
178$\times$ when compared to Geant4, and is approximately 3.3$\times$ faster
than \acs{adept}. For a serial \ac{cpu} run, \celeritas is about 1.6$\times$
faster than Geant4.

\begin{figure}
    \centering
    \includegraphics[width=0.7\textwidth]{figs/testem3_edep.pdf}
    \caption{Mean energy deposition per layer, represented by integer numbers
    in the $x$ axis. Each layer is composed by a pair of absorber and gap made
    of Pb and liquid Ar.}
    \label{fig:testem3-edep}
\end{figure}

\begin{table}
    \caption{Comparison between \celeritas, \acs{adept}, and Geant4 performances
    in events per second. The speedup factor is calculated with respect to
    Geant4.}
    \label{tab:testem3-performance}
    \centering
    \begin{tabular}{llrr}
        \toprule
        Application & Execution & Performance [events/s] & Speedup factor\\
        \midrule
        Geant4      & \ac{cpu} (serial) & 11.4   & 1.0\\
        \acs{adept} & \ac{gpu}          & 618.5  & 54.0\\
        \celeritas  & \ac{gpu}          & 2040.9 & 178.2\\
        \celeritas  & \ac{cpu} (serial) & 18.5   & 1.6\\
        \bottomrule
    \end{tabular}
  \end{table}
