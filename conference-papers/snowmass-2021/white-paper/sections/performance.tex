%%---------------------------------------------------------------------------%%
% performance.tex
%%---------------------------------------------------------------------------%%
\section{\celeritas performance}

\celcomment{Discuss comparisons with AdePT and if they should go in the white
paper}

% - - - - - - - - - - - - - - - - - - - - - - - - - - - - - - - - - - - - - - %
\subsection{Performance metrics}

The range of detector geometry complexities, primary event types, and amount of output, there is no general case that provides a simple
performance number such as a simulation rate in events per second. Therefore, our ultimate performance metric
will be based on a set of real-world \acs{hep} detector workflow use cases. \celeritas must run the targeted models at the same
fidelity as the current state of the art but in much less time. Two \acp{fom}
will be used to measure success.

On \ac{lcf} hardware for the next decade, \acp{gpu} will provide the bulk of the
compute capability, so one critical performance metric is the ratio of the
runtime of \celeritas using \acp{gpu} to using only the \acp{cpu} of a given
machine. This \ac{fom} represents the ability of \celeritas to effectively use
the hardware that is available and must be sufficiently high to justify running
on \ac{lcf} resources. We will target a factor of
$160\times$ for a single \ac{gpu} to a single \ac{cpu}, which is the relative
\ac{gpu}/\ac{cpu} performance of the Shift \ac{mc} transport code \cite{hamilton_continuous-energy_2019}.

A second performance metric is critical to the programmatic viability of
\celeritas in the broader \ac{hep} community: the work done for the same amount
of cost in power consumption and hardware, using Geant4 on \ac{cpu} as a
baseline. This \ac{fom} is the motivation for \ac{hep} workflows to adapt to
using \celeritas for \ac{mc} transport. The Geant4 team supposes that a factor
of two speedup resulting from ``adiabatic improvements'' to their code is not
outside the realm of possibility \cite{marc_verderi_geant4_2021}, so a $2
\times$ cost improvement of \celeritas runtime over Geant4 runtime is our second
target. Assuming that electricity consumption (and waste heat disposal) are the
primary constraints for independent \ac{hep} computing centers, this factor
should be evaluated by comparing the performance of Geant4 on \acp{cpu} with a
comparable power requirement as \celeritas on \acp{gpu}. At the present time,
for example, that comparison might be for an AMD 3rd Gen EPYC Processor power
(\SI{280}{\watt} for 64 cores) to a PCIe \nvidia A100 (\SI{250}{\watt} for 108
symmetric multiprocessors).

Meeting these \acp{fom} with Geant4-equivalent physics capabilities and
providing a solid user interface and \ac{io} integration to \ac{hep} workflows
will be the ultimate confirmation that \celeritas is a viable option for
\ac{hep} experiments. At this point \celeritas will be ideal for execution on
\ac{doe} \acp{lcf} and will be sufficiently fast on its own merits to motivate
independent adoption on capacity systems.

% - - - - - - - - - - - - - - - - - - - - - - - - - - - - - - - - - - - - - - %
\subsection{Test problem definition}

% - - - - - - - - - - - - - - - - - - - - - - - - - - - - - - - - - - - - - - %
\subsection{Preliminary results}
