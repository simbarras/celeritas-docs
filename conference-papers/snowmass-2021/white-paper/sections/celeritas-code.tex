%%---------------------------------------------------------------------------%%
% celeritas-code.tex
%%---------------------------------------------------------------------------%%
\section{\celeritas code}

%% - - - - - - - - - - - - - - - - - - - - - - - - - - - - - - - - - - - - - %%
%% Code Architecture
%% - - - - - - - - - - - - - - - - - - - - - - - - - - - - - - - - - - - - - %%
\subsection{Code Architecture}

A detailed description of the \celeritas code architecture is given in
\textcite{johnson_2021}. The code base (Fig.~\ref{fig:celeritas-code-base})
relies on external dependencies for key capabilities that will be discussed over
the next sections.
%%
\begin{figure}
  \centering
  \includegraphics[scale=.6]{figs/software-deps.pdf}
  \caption{\celeritas code base (white) and its existing (gray) and proposed
    (magenta) third-party dependencies, both required (solid lines) and optional
    (dashed lines).}
  \label{fig:celeritas-code-base}
\end{figure}
%%

\celeritas supports intra-node concurrency on multi-core architectures through
\acs{openmp} and on \nvidia \acp{gpu} using \cuda, which we plan to supplement
with a programming model for vendor-independent portability.  Internode
parallelism is currently implemented using the \ac{mpi} communication library
through a domain replication model in which particle events are decomposed
across \ac{mpi} ranks.

Like other \ac{gpu}-enabled \ac{mc} transport codes such as Shift
\cite{pandya_implementation_2016,hamilton_multigroup_2018,
hamilton_continuous-energy_2019,hamilton_domain_2022}, the low-level component
code used by transport kernels is designed so that each particle track
corresponds to a single thread. There is no cooperation between individual
threads, facilitating the dual host/device annotation of most of \celeritas.
\celeritas also uses a modular programming approach based on composition rather
than inheritance in order to accommodate device-based architectures, which have
poor support for runtime polymorphism.  Furthermore, the use of simple classes
that define function-object interfaces enables lambda semantics for kernel
launches. These features will simplify the adaptation to a performance
portability model.

The \celeritas programming model uses the \ac{dop} paradigm \cite{dop_2022} to
facilitate platform portability, improve memory access patterns, and accelerate
development. \ac{dop} separates execution code from data, and as part of this
model \celeritas carefully partitions immutable, shared ``parameter'' data from
dynamic thread-local ``state'' data. Object-oriented design patterns encapsulate
the data storage implementation, temporarily combining parameter and state data
into ``view'' classes.  Higher-level classes use composition to combine the data
from the multiple entities that comprise a particle track's complete state.

In the first 1.5 years of \celeritas' development, nine \ac{gpu}-compatible
physics models (Table~\ref{tab:em-physics}) have been implemented. This shows
\ac{dop} to be highly effective for development on heterogeneous architectures
that have independent \emph{memory spaces} between which data must be
transferred. One challenge faced by \ac{mc} physics application codes is the
ubiquity of complicated \emph{heterogeneous} data structures needed for
tabulated physics and particle data, as opposed to the simpler
\emph{homogeneous} data layouts required by deterministic numerical solvers. A
novel programming model in \celeritas enables the composition of new, deep data
types (e.g., material properties) that are required by geometric and physics
operations during the transport loop without fragmenting the underlying data
layout on device.

One requirement for transporting particles in \ac{em} showers is the efficient
allocation and construction of secondary particles during a physics interaction.
On \acp{gpu}, managing dynamic allocations efficiently is a significant
challenge.  To enable runtime dynamic allocation of secondary particles, we have
developed a function-like stack allocator that accesses a large on-device
allocated array with a fixed capacity and uses an atomic addition to
unambiguously reserve one or more items in the array. The final aspect of
\ac{gpu}-based secondary allocation is how to gracefully handle an out-of-memory
condition without crashing the simulation \emph{or} invalidating its
reproducibility. A novel algorithm in \celeritas guarantees robustness when
allocating secondaries, which we will extend to guarantee complete
reproducibility of \ac{hep} workflow results.

The \celeritas code architecture just summarized is designed to enable \ipl{1}
performance portability, \ipl{2} implementation of new physics models and
processes, \ipl{3} optimal geometric tracking, \ipl{4} optimization of the
particle transport algorithm in the presence of external \ac{em} fields, and
\ipl{5} addition of scoring and \ac{ui} necessary to meet \acs{hep} detector
simulation requirements.

%% - - - - - - - - - - - - - - - - - - - - - - - - - - - - - - - - - - - - - %%
\subsubsection{Platform portability}

The next generation platforms at the \ac{olcf}, \ac{alcf}, and \ac{nersc} will
each feature different \ac{gpu}-based architectures as shown in
Table~\ref{tab:lcf-arch}. Although \celeritas currently relies on \cuda, its
highly modular data management design allows compilation and testing on standard
multicore architectures. This strategy supports the straightforward
implementation of programming models for platform portability, with significant
advances using \acs{hip} being underway.
%%
\begin{table}
  \caption{Exascale architectures and native programming models at \acs{doe}
  \acsp{lcf}.}
  \label{tab:lcf-arch}
  \centering%
  \begin{tabular}{lllll}\toprule
      %%
    Center & Machine & Integrator & \acsp{gpu} & Native Programming
    Model\\\midrule
        %%
    \acs{olcf} & Frontier & HPE & \amd & \acs{hip}\textsuperscript{a}\\
        %%
    \acs{alcf} & Aurora & \intel & \intel &
    \oneapi/DP\Cpp/SYCL\textsuperscript{b}\\
        %%
    \acs{nersc} & Perlmutter & HPE & \nvidia &
    \cuda\hspace{-.4em}\textsuperscript{c}\\
    \bottomrule
        %%
    \multicolumn{5}{l}{\footnotesize
      \textsuperscript{a}\url{https://github.com/ROCm-Developer-Tools/HIP} }\\
    \multicolumn{5}{l}{\footnotesize
      \textsuperscript{b}\url{https://software.intel.com} }\\
    \multicolumn{5}{l}{\footnotesize
      \textsuperscript{c}\url{https://docs.nvidia.com/cuda/cuda-runtime-api/index.html}
      }\\
  \end{tabular}
\end{table}

\textcite{evans_survey_2021} surveyed the various \ac{gpu} programming models
employed by applications within the \ac{ecp} and found Kokkos
\cite{CarterEdwards20143202}, \acs{hip}, and \acs{openmp} to be the most
commonly employed programming models.  Each of these models has pros and cons:
some models are not yet supported on all \ac{gpu} architectures, and experience
in \ac{ecp} has shown that performance can vary dramatically depending on the
maturity of both the software stack and the underlying hardware. This is also a
time of significant change in the \Cpp language itself, particularly with
respect to concurrency support via Standard Library algorithms. Combined with
the increasing adoption of LLVM for \Cpp compiler development, we anticipate
that more vendors will provide \ac{gpu}-based concurrency \acp{api} through \Cpp
Standard Library constructs within the next \numrange[range-phrase={ to
}]{5}{10} years, providing yet another possible means of achieving platform
portability.

The \celeritas team will evaluate and subsequently choose a programming model to
provide portable execution across all major \ac{gpu} vendors as well as
traditional multicore \acp{cpu}. This evaluation will be based on achievable
performance across multiple \ac{gpu} architectures, ease of integration into
\celeritas workflows, sustainability, availability of multiple implementations,
and ability to perform platform-specific tuning. We also plan to engage the
\ac{hep} \ac{cce} \ac{pps} working group to ensure our strategy is in line with
other efforts in the \ac{hep} community. At the present time, our nominal
performance portability plan for \celeritas is to utilize one of the \Cpp-based
programming models (Kokkos, SYCL, or \Cpp standard library execution policies);
however, we recognize that these models are rapidly changing as is compiler
support, and thus a proper evaluation is necessary before making a decision.


%% - - - - - - - - - - - - - - - - - - - - - - - - - - - - - - - - - - - - - %%
%% Physics
%% - - - - - - - - - - - - - - - - - - - - - - - - - - - - - - - - - - - - - %%
\subsection{Physics}

The key physics component in \celeritas is a \emph{process}, which defines an
observed physical phenomenon such as the photoelectric effect or bremsstrahlung.
Each process is implemented as one or more models that each mathematically
describe or approximate the process in a given energy regime.

The initial implementation in \celeritas targets \ac{em} physics between
\SI{100}{eV} and \SI{100}{TeV} for photons, electrons, and positrons. This
minimal set of capabilities, with physical processes and associated numerical
models itemized in Table~\ref{tab:em-physics}, is necessary to generate
realistic simulations of \ac{em} showers and demonstrate key characteristics of
a full-featured transport loop.
%%
\begin{table}
  \caption{Current status of \celeritas \acs{em} physics. The initial
  implementation ($\gamma$ and $e^\pm$) is almost complete, and muon physics is
  in its initial stage. Particle symbols are defined in
  \textcite{tanabashi_review_2018}.}
  \label{tab:em-physics}
  \centering
  \begin{tabular}{clll}\toprule Particle & Process & Model(s) & Status\\
    \midrule
    %% gamma
    \multirow{4}{*}{$\gamma$}
    & photon conversion & Bethe--Heitler & implemented\\
    & Compton scattering & Klein--Nishina & verified\\
    & photoelectric effect & Livermore & implemented\\
    & Rayleigh scattering & Livermore & implemented\\
    \midrule
    %% e^+-
    \multirow{4}{*}{$e^\pm$}
    & ionization & M\o{}ller--Bhabha & implemented\\
    & bremsstrahlung & Seltzer--Berger, relativistic & implemented\\
    & pair annihilation & EPlusGG & implemented\\
    & multiple scattering & Urban, WentzelVI & in progress\\
    \midrule
    $\mu^\pm$ & muon bremsstrahlung & Muon Bremsstrahlung & implemented\\
    \bottomrule
  \end{tabular}
\end{table}
%%
These already implemented characteristics include \ipl{1} material-dependent
physical properties, \ipl{2} continuous slowing down in matter for charged
particles, \ipl{3} selecting discrete interactions among competing processes,
\ipl{4} scattering or absorbing particles during an interaction, \ipl{5}
emitting secondary particles, and \ipl{6} applying energy cutoffs to cull
low-energy photons and electrons.

The physics implementation in \celeritas focuses on maximizing work done in
parallel. For example, all particle types use tabulated discrete interaction
cross sections calculated simultaneously in a single kernel. The primary
deviation from this rule is that each model of a discrete process launches an
independent kernel that applies only to tracks undergoing an interaction with
that process. This set of kernel launches is performed polymorphically from
\ac{cpu} host code, allowing arbitrary noninvasive extensions to \celeritas
physics.

In order to meet the detector simulation requirements for \acs{hep} experiments,
\celeritas physics will be expanded from its initial \ac{em} prototype to a full
set of particles with decay and hadronic physics.  A complete list of the
required physics processes and particles is shown in
Table~\ref{tab:proposed-physics}, where only processes are explicit as model
separation will be determined based on performance and/or code maintainability.
%%
\begin{table}
  \caption{Proposed physics development in \celeritas.}
  \label{tab:proposed-physics}
  \centering
  \begin{tabular}{llll}
    \toprule
    Physics & Process & Particle(s)\\
    \midrule
    %% EM
    \multirow{10}{*}{\acs{em}} & photon conversion & $\gamma$\\
    & pair annihilation & $e^\pm$\\
    & photoelectric effect& $\gamma$\\
    & ionization & charged leptons, hadrons, and ions\\
    & bremsstrahlung & charged leptons and hadrons\\
    & Rayleigh scattering & $\gamma$\\
    & Compton scattering & $\gamma$\\
    & Coulomb scattering & charged leptons, hadrons\\
    & multiple scattering & charged leptons, hadrons\\
    & continuous energy loss & charged leptons, hadrons, and ions\\
    \midrule
    %% Decay
    \multirow{3}{*}{Decay}
    & two body decay & $\mu^\pm$, $\tau^\pm$, hadrons\\
    & three body decay & $\mu^\pm$, $\tau^\pm$, hadrons\\
    & n-body decay & $\mu^\pm$, $\tau^\pm$, hadrons\\
    \midrule
    %% Hadronic
    \multirow{6}{*}{Hadronic}
    & photon-nucleus & $\gamma$ \\
    & lepton-nucleus & leptons \\
    & nucleon-nucleon & $p$, $n$\\
    & hadron-nucleon & hadrons\\
    & hadron-nucleus & hadrons\\
    & nucleus-nucleus & hadrons\\
    \bottomrule
  \end{tabular}
\end{table}
%%

Geant4 manages cross sections by a combination of importing experimental values
from external databases and programmatic implementations of theoretical models.
Most of the data used in the simulation is pre-tabulated during initialization,
with some models calculating elemental cross sections at runtime. Unlike Geant4,
\celeritas currently does not have a fully-fledged system to load and
pre-tabulate cross sections from external databases. Therefore, cross section
tables are imported from Geant4 via an external application---the Geant4
exporter in Fig.~\ref{fig:celeritas-code-base}. This application loads the
problem geometry via a \acs{gdml} file, initializes Geant4, and stores all the
pre-calculated cross sections for all the available physics processes and models
into a ROOT file. This file is then used by a \celeritas application to load the
data to host and device at initialization time.

%% - - - - - - - - - - - - - - - - - - - - - - - - - - - - - - - - - - - - - %%
%% Geometry
%% - - - - - - - - - - - - - - - - - - - - - - - - - - - - - - - - - - - - - %%
\subsection{Geometry and \ac{em} fields}

Accurate simulation of \ac{hep} detector output requires a highly detailed model
of the detector apparatus and nearby components. \ac{mc} transport requires that
the model geometry be well-defined at every point in space, requiring a model
that is ``watertight,'' or heuristics for recovering from inconsistencies (e.g.,
overlapping or missing regions), or both. Since particles undergo collisions and
charged particles constantly change direction as they move through the magnetic
fields in the detector, traditional straight-line ray tracing is necessary but
not sufficient to correctly navigate the geometry.

The \acs{vecgeom} library supports navigation through Geant4-defined detector
geometries in \ac{cpu}-only code and on \cuda-enabled devices
\cite{apostolakis_towards_2015} and is the initial and primary geometry package
in \celeritas. Figure~\ref{fig:vecgeom-cms} is a representation of the \ac{cms}
geometry traced in parallel on \ac{gpu} using the \celeritas particle tracking
interface to \acs{vecgeom}.
%%
\begin{figure}
  \centering%
  \begin{subfigure}{3in}%
    \centering%
    \includegraphics[height=2in]{figs/cms-xz.pdf}%
    \caption{$y=0$ [cm]}%
  \end{subfigure}%
  \begin{subfigure}{3in}%
    \centering%
    \includegraphics[height=2in]{figs/cms-xy.pdf}%
    \caption{$z=0$ [cm]}%
  \end{subfigure}
  \caption{On-device \acs{vecgeom} ray traces of the \acs{cms} generated with
the \celeritas ray trace demonstration app.}
\label{fig:vecgeom-cms}
\end{figure}

A new, alternate geometry in \celeritas provides a test bed for experimenting
with platform-portable navigation that uses fundamentally different algorithms
from \acs{vecgeom}. This implementation, \ac{orange}, is an initial \ac{gpu}
port of the new modernized geometry used by the SCALE nuclear engineering code
suite to model complex multi-level nuclear reactor and neutral particle
shielding problems~\cite{scale}. The \celeritas port uses the Collection
paradigm
%
\celcomment{(make sure this definition is explained)}
%
to store quadric surface representations and define cell volumes as \ac{csg}
combinations of those surfaces. At present the \ac{gpu} prototype implementation
supports only a single geometry level, but the extension of this implementation
to the full capabilities available on \ac{cpu} will be relatively
straightforward. We are actively collaborating with the \acs{vecgeom} group at
\acs{cern} to research how \ac{orange} and its methodology could power the next
generation of Geant4 tracking on \ac{gpu}.

\celeritas supports on-device propagation of particles through arbitrary
magnetic fields. Using a \Cpp-based template system for high-performance,
platform-independent extensibility, it allows different integration algorithms
for different user-defined fields.

%% - - - - - - - - - - - - - - - - - - - - - - - - - - - - - - - - - - - - - %%
%% User-interface and I/O
%% - - - - - - - - - - - - - - - - - - - - - - - - - - - - - - - - - - - - - %%
\subsection{User-interface and \ac{io}}

The key challenge in providing user interfaces is developing a framework that
supports integration into existing experimental \ac{mc} simulation sequences
that \ipl{1} provides the minimum required code barrier for incorporation and
\ipl{2} preserves performance.  Existing experimental frameworks are built on
the Geant4 toolkit, which provides user actions that allow users to control the
program and data flow at the level of run, event, track, and step. In Geant4
scoring is done using dedicated stepping actions in which information from the
sensitive detector volumes is accessible through callback semantics into these
parts of the simulation.  This approach provides great user flexibility at the
cost of higher computational overhead and increased system complexity.
Furthermore, callback functions will not work in accelerator code because it is
not possible to call host functions in the middle of device kernel execution.

\celeritas will not operate as a toolkit as Geant4 does, since this would leave
many implementation decisions to the end-user, hindering performance. Therefore,
to address the purely technical challenges of supporting experimental workflows,
we will implement an \ac{api} through which clients can specify geometric
regions for scoring and \ac{mc} particle data. Using this \ac{api} the desired
scoring data can be processed on the host at runtime, and the necessary data
fields for tallies can then be configured for execution in kernel code on the
device.  The most efficient interface would fully occupy the device by executing
many events concurrently.  However, this is not the way that experimental
workflows are currently configured, in which events are executed independently
on each thread.

As \ac{hpc} evolved to make use of heterogeneous architectures, \ac{io} became
one of the critical performance bottlenecks of many \ac{hpc} applications, being
a main concern for \celeritas. \ac{hep} detector simulations produce large
volumes of data, with many millions of particle tracks and detector scoring
regions having to be recorded. Thus, the need to optimize the data movement
between host and device and manage parallel \ac{io} requests is paramount.
Furthermore, to be compatible with \ac{hep} experimental workflows, \celeritas
needs to be integrated with ROOT. To mitigate these problems, we will cross
collaborate with \ac{doe}'s \ac{rapids} team to make \ac{adios} the internal
\ac{io} \ac{api} of \celeritas, as it is highly optimized for heterogeneous
architectures. The ultimate goal is find optimal strategies to mitigate \ac{io}
performance issues, and integrate \ac{adios} with ROOT for full interoperability
with \ac{hep} workflows. This colaboration with \ac{rapids} will also explore
data visualization tools and event filtering, allowing users to visualize,
validate, and debug the generated data before launching production campaigns.
